\preface % Note: \preface JANGAN DIHAPUS!


Segala puji dan syukur kehadiran Allah SWT yang telah melimpahkan rahmat dan hidayah-Nya kepada kita semua, sehingga penulis dapat menyelesaikan penulisan Proposal Penelitian yang berjudul \textbf{“ANALISIS AKUISISI DATA DAN MOSAIK CITRA 
RESOLUSI TINGGI KAWASAN KAMPUS II 
UNIVERSITAS SYIAH KUALA DI KECAMATAN 
MESJID RAYA, ACEH BESAR”} yang telah dapat diselesaikan sesuai rencana. Penulis banyak mendapatkan berbagai pengarahan, bimbingan, dan bantuan dari berbagai pihak. Oleh karena itu, melalui tulisan ini penulis mengucapkan rasa terima kasih kepada:

\begin{enumerate}
	\item{Ayah dan Ibu sebagai kedua orang tua penulis yang senantiasa selalu mendukung aktivitas dan kegiatan yang penulis lakukan baik secara moral maupun material serta menjadi motivasi terbesar bagi penulis untuk menyelesaikan Proposal Penelitian ini.}
    \item{Bapak Prof. Dr. Taufik Fuadi Abidin, S.Si, M.Tech selaku Dekan Fakultas MIPA Universitas Syiah Kuala} 
	\item{Bapak Dr. Nizamuddin, M.Info.Sc. selaku Dosen Pembimbing I yang telah banyak memberikan bimbingan dan arahan kepada penulis, sehingga penulis dapat menyelesaikan Proposal Penelitian ini.}
	\item{Bapak Ardiansyah, BSEE.,M.Sc selaku Dosen Pembimbing II yang telah banyak memberikan bimbingan dan arahan kepada penulis, sehingga penulis dapat menyelesaikan Proposal Penelitian ini.}
	\item{Bapak Arie Budiansyah, ST., M.Eng., selaku Dosen Wali.}
	\item {Alim Misbullah, S.Si., M.S., selaku Ketua Program Studi Informatika.}
	\item{Seluruh Dosen di Jurusan Informatika Fakultas MIPA atas ilmu dan didikannya selama perkuliahan.}
	\item{Sahabat dan teman-teman seperjuangan Jurusan Informatika USK 2020 lainnya.}
\end{enumerate}

%\vspace{1.5cm}

Penulis juga menyadari segala ketidaksempurnaan yang terdapat didalamnya baik dari segi materi, cara, ataupun bahasa yang disajikan. Seiring dengan ini penulis mengharapkan kritik dan saran dari pembaca yang sifatnya dapat berguna untuk kesempurnaan Proposal Penelitian ini. Harapan penulis semoga tulisan ini dapat bermanfaat bagi banyak pihak dan untuk perkembangan ilmu pengetahuan.

\vspace{1cm}


\begin{tabular}{p{7.5cm}l}
	&Banda Aceh, 22 April 2024\\
	&\\
	&\\
	&\multirow{1.5}{7.5cm}{\underline{Muhammad Ghaficky Renku}} \\ 
	&NPM. 2008107010046 \\
\end{tabular}
